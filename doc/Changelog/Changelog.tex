\documentclass[a4paper,10pt]{article}

\usepackage{ngerman}
\usepackage[utf8]{inputenc}
\usepackage{amsmath}
\usepackage{amssymb}
\usepackage{multirow}


\usepackage[left=2.2 cm,right=2.2 cm,top=1.5cm,bottom=2.0cm]{geometry}


\begin{document}
\begin{center}
\section*{KUR2}
\subsection*{Versionshistorie}
\end{center}

\hrule
\vspace{.5cm}

\subsubsection*{Version 1.1:}
\begin{itemize}
\item \textbf{Feature:} Größe des Speichers kann durch das Argument \textit{memory=<Speichergröße>} angepasst werden
\item \textbf{Bugfix:} Programm hängte sich beim Zugriff auf nicht existente Speicherzellen auf. Jetzt gibt es eine Fehlermeldung aus, bevor es sich beendet.
\end{itemize}

\subsubsection*{Version 1.2:}
\begin{itemize}
	\item \textbf{Misc:} Standardquelle für Programme ist jetzt \textit{res/progs/defaultProgram.txt}
	\item \textbf{Feature:} Optionen für \textit{Programmkontrolle}:
		\begin{itemize}
			\item \textit{Programm laden} erlaubt gezielt die Quelldatei zu bestimmen (Nicht mehr nur über \textit{res/progs/testProgram.txt})
			\item \textit{Maschine zurücksetzen} erlaubt das aktuelle Programm erneut auszuführen, ohne das Programm neu starten zu müssen
		\end{itemize}
	\item \textbf{Feature:} Optionen für \textit{stepping}:
		\begin{itemize}
			\item Einzelschritt ausführen
			\item Mikroinstruktion ausführen
			\item Maschinenbefehl ausführen
		\end{itemize}
	\item \textbf{Misc:} \textit{Look and Feel} wird vom OS bestimmt
\end{itemize}

\end{document}